\documentclass{article}

\usepackage[T2A]{fontenc}
\usepackage{amsmath}

\newcommand\hr{\par\vspace{-.5\ht\strutbox}\noindent\hrulefill\par}

\begin{document}
\thispagestyle{empty}

\begin{scriptsize} § 2. ОСНОВНЫЕ ПРАВИЛА ДИФФЕРЕНЦИРОВАНИЯ \end{scriptsize} 
\hfill 183
\bigskip

что представляется вполне естественным, если символ $\frac{d z}{d y}$ или $\frac{d y}{d x}$ рассматривать не как единый, а как отношение $d z$ к $d y$ или, соответственно, $d y$ к $d x$.
Возникающая в связи с этим идея доказательства состоит в том, чтобы рассмотреть разностное отношение
$$
\frac{\Delta z}{\Delta x}=\frac{\Delta z}{\Delta y} \cdot \frac{\Delta y}{\Delta x}
$$
и затем перейти к пределу при $\Delta x \rightarrow 0$. Трудность, которая тут появляется (и с которой нам тоже отчасти пришлось считаться!), состоит в том, что $\Delta y$ может быть нулем, даже если $\Delta x \neq 0$.

СЛЕДСТВИЕ 2. Если имеется композиция $\left(f_n \circ \ldots \circ f_1\right)(x)$ дифференцируемых функциŭ $y_1=f_1(x), \ldots, y_n=f_n\left(y_{n-1}\right)$, mo
$$
\left(f_n \circ \ldots \circ f_1\right)^{\prime}(x)=f_n^{\prime}\left(y_{n-1}\right) f_{n-1}^{\prime}\left(y_{n-2}\right) \ldots f_1^{\prime}(x)
$$

\textbf{<} При $n=1$ утверждение очевидно. \mbox{}\hfill

Если оно справедливо для некоторого $n \in \mathbb{N}$, то из теоремы 2 следует, что оно справедливо также для $n+1$, т. е. по принципу индукции установлено, что оно справедливо для любого $n \in \mathbb{N}$. \textbf{>}

ПРИМЕР 5. Покажем, что при $\alpha \in \mathbb{R}$ в области $x>0$ имеем $\frac{d x^a}{d x}=\alpha x^{\alpha-1}$, T. e. $d x^\alpha=\alpha x^{\alpha-1} d x$, и
$$
(x+h)^\alpha-x^\alpha=\alpha x^{\alpha-1} h+o(h) \quad \text { при } h \rightarrow 0 .
$$
\textbf{<} Запишем $x^\alpha=e^{\alpha \ln x}$ и применим доказанную теорему с учетом результатов примеров 9 и 11 из \& 1 и пункта b) теоремы 1.
Пусть $g(y)=e^y$ и $y=f(x)=\alpha \ln x$. Тогда $x^\alpha=(g \circ f)(x)$ и
$$
(g \circ f)^{\prime}(x)=g^{\prime}(y) \cdot f^{\prime}(x)=e^y \cdot \frac{\alpha}{x}=e^{\alpha \ln x} \cdot \frac{\alpha}{x}=x^\alpha \cdot \frac{\alpha}{x}=\alpha x^{\alpha-1} . \textbf{>}
$$

ПРИМЕР 6. Производная от логарифма модуля дифференцируемой функции часто называется логарифмической производной.

Поскольку $F(x)=\ln |f(x)|=(\ln \circ|| \circ f)(x)$, то в силу результата примера 11 из $\S 1 F^{\prime}(x)=(\ln |f|)^{\prime}(x)=\frac{f^{\prime}(x)}{f(x)}$.

Таким образом,
$$
d(\ln |f|)(x)=\frac{f^{\prime}(x)}{f(x)} d x=\frac{d f(x)}{f(x)}
$$

ПРИМЕР 7. Абсолютная и относительная погрешности значения дифференцируемой функции, вьзваннье погрешностями в задании аргумента. Если функция $f$ дифференцируема в точке $x$, то
$$
f(x+h)-f(x)=f^{\prime}(x) h+\alpha(x ; h)
$$
где $\alpha(x ; h)=o(h)$ при $h \rightarrow 0$.

Таким образом, если при вычислении значения $f(x)$ функции аргумент $x$ определен с абсолютной погрешностью $h$, то вызванная этой погрешно-


\newpage
\thispagestyle{empty}

\begin{scriptsize}ГЛ. v. ДИФФЕРЕНЦИАЛЬНОЕ ИСЧИСЛЕНИЕ  \end{scriptsize} \hfill 184

\bigskip
стью абсолютная погрешность $|f(x+h)-f(x)|$ в значении функции при достаточно малых $h$ может быть заменена модулем значения дифференциала $|d f(x) h|=\left|f^{\prime}(x) h\right|$ на смещении $h$.

Тогда относительная погрешность может быть вычислена как отношение $\frac{\left|f^{\prime}(x) h\right|}{|f(x)|}=\frac{|d f(x) h|}{|f(x)|}$ или как модуль произведения $\left|\frac{f^{\prime}(x)}{f(x)}\right||h|$ логарифмической производной функции на величину абсолютной погрешности аргумента.

Заметим, кстати, что если $f(x)=\ln x$, то $d \ln x=\frac{d x}{x}$ и абсолютная погрешность в определении значения логарифма равна относительной погрешности в определении аргумента. Это обстоятельство прекрасно используется, например, в логарифмической линейке (и многих других приборах с неравномерным масштабом шкал). А именно, представим себе, что с каждой точкой числовой оси, лежащей правее нуля, мы связали ее координату у и записали ее над точкой, а под этой точкой записали число $x=e^y$. Тогда $y=$ $=\ln x$. Одна и та же числовая полуось оказалась наделенной одной равномерной шкалой у и одной неравномерной (ее называют логарифмической) шкалой $x$. Чтобы найти $\ln x$, надо установить визир на числе $x$ и прочитать наверху соответствующее число $y$. Поскольку точность установки визира на какую-то точку не зависит от числа $x$ или $y$, ей отвечающего, и измеряется некоторой величиной $\Delta y$ (длиной отрезка возможного уклонения) в равномерной шкале, то при определении по числу $x$ его логарифма $y$ мы будем иметь примерно одну и ту же абсолютную погрешность, а при определении числа по его логарифму будем иметь примерно одинаковую относительную погрешность во всех частях шкалы.

ПРИМЕР 8. Продифференцируем функцию $u(x)^{\nu(x)}$, где $u(x)$ и $v(x)-$ дифференцируемые функции и $u(x)>0$. Запишем $u(x)^{\nu(x)}=e^{\nu(x) \ln u(x)}$ и воспользуемся следствием 2
$$
\begin{aligned}
& \frac{d e^{v(x) \ln u(x)}}{d x}=e^{v(x) \ln u(x)}\left(v^{\prime}(x) \ln u(x)+v(x) \frac{u^{\prime}(x)}{u(x)}\right)= \\
&=u(x)^{v(x)} \cdot v^{\prime}(x) \ln u(x)+v(x) u(x)^{v(x)-1} \cdot u^{\prime}(x) .
\end{aligned}
$$

\textbf{3. Дифференцирование обратной функции}

ТЕОРЕМА 3 (теорема о производной обратной функции). Пусть функции $f: X \rightarrow Y, f^{-1}: Y \rightarrow X$ взаимно обратны и непрерывны в точках $x_0 \in X$ u $f\left(x_0\right)=y_0 \in Y$ соответственно. Если функция $f$ дифференцируема в точке $x_0$ и $f^{\prime}\left(x_0\right) \neq 0$, то функция $f^{-1}$ также дифференцируема в точке $y_0$, причем
$$
\left(f^{-1}\right)^{\prime}\left(y_0\right)=\left(f^{\prime}\left(x_0\right)\right)^{-1}
$$
< Поскольку функции $f: X \rightarrow Y, f^{-1}: Y \rightarrow X$ взаимно обратны, то величины $f(x)-f\left(x_0\right), f^{-1}(y)-f^{-1}\left(y_0\right)$ при $y=f(x)$ не обращаются в нуль, если $x \neq$ $\neq x_0$. Из непрерывности $f$ в $x_0$ и $f^{-1}$ в $y_0$ можно, кроме того, заключить, что $\left(X \ni x \rightarrow x_0\right) \Leftrightarrow\left(Y \ni y \rightarrow y_0\right)$. Используя теперь теорему о пределе композиции


\newpage
\thispagestyle{empty}

\begin{scriptsize}§ 2. ОСНОВНЫЕ ПРАВИЛА ДИФФЕРЕНЦИРОВАНИЯ  \end{scriptsize} \mbox{}\hfill 185

\bigskip
функций и арифметические свойства предела, находим
$$
\lim _{Y \ni y \rightarrow y_0} \frac{f^{-1}(y)-f^{-1}\left(y_0\right)}{y-y_0}=\lim _{x \ni x \rightarrow x_0} \frac{x-x_0}{f(x)-f\left(x_0\right)}=\lim _{x \ni x \rightarrow x_0} \frac{1}{\left(\frac{f(x)-f\left(x_0\right)}{x-x_0}\right)}=\frac{1}{f^{\prime}\left(x_0\right)} . >
$$
Таким образом, показано, что в точке $y_0$ функция $f^{-1}: Y \rightarrow X$ имеет производную и
$$
\left(f^{-1}\right)^{\prime}\left(y_0\right)=\left(f^{\prime}\left(x_0\right)\right)^{-1} \text {. }
$$
ЗАМЕЧАНИЕ 1. Если бы нам заранее было известно, что функция $f^{-1}$ дифференцируема в точке $y_0$, то из тождества $\left(f^{-1} \circ f\right)(x)=x$ по теореме о дифференцировании композиции функций мы сразу же нашли бы, что $\left(f^{-1}\right)^{\prime}\left(y_0\right) \cdot f^{\prime}\left(x_0\right)=1$

ЗАМЕЧАНИЕ 2. Условие $f^{\prime}\left(x_0\right) \neq 0$, очевидно, равносильно тому, что отображение $h \mapsto f^{\prime}\left(x_0\right) h$, осуществляемое дифференциалом $d f\left(x_0\right): T \mathbb{R}\left(x_0\right) \rightarrow$ $\rightarrow T \mathbb{R}\left(y_0\right)$, имеет обратное отображение $\left[d f\left(x_0\right)\right]^{-1}: T \mathbb{R}\left(y_0\right) \rightarrow T \mathbb{R}\left(x_0\right)$, задаваемое формулой $\tau \mapsto\left(f^{\prime}\left(x_0\right)\right)^{-1} \tau$.

Значит, в терминах дифференциалов вторую фразу формулировки теоремы 3 можно было бы записать следующим образом:

Если функция $f$ дифференцируема в точке $x_0$ и в этой точке ее дифференциал df $\left(x_0\right): T \mathbb{R}\left(x_0\right) \rightarrow T \mathbb{R}\left(y_0\right)$ обратим, то дифференциал функции $f^{-1}$, обратной $\kappa f$, существует в точке $y_0=f\left(x_0\right)$ и является отображением
$$
d f^{-1}\left(y_0\right)=\left[d f\left(x_0\right)\right]^{-1}: T R\left(y_0\right) \rightarrow T R\left(x_0\right),
$$
обратным к отображению $d f\left(x_0\right): T \mathbb{R}\left(x_0\right) \rightarrow T \mathbb{R}\left(y_0\right)$.

ПРИМЕР 9. Покажем, что $\arcsin ^{\prime} y=\frac{1}{\sqrt{1-y^2}}$ при $|y|<1$.
Функции $\sin :[-\pi / 2, \pi / 2] \rightarrow[-1,1]$ и arcsin: $[-1,1] \rightarrow[-\pi / 2, \pi / 2]$ взаимно обратны и непрерывны (см. гл. IV, §2, пример 8), причем $\sin ^{\prime} x=\cos x \neq 0$, если $|x|<\pi / 2$. При $|x|<\pi / 2$ для значений $y=\sin x$ имеем $|y|<1$.
Таким образом, по теореме 3
$$
\arcsin ^{\prime} y=\frac{1}{\sin ^{\prime} x}=\frac{1}{\cos x}=\frac{1}{\sqrt{1-\sin ^2 x}}=\frac{1}{\sqrt{1-y^2}} .
$$
Знак перед радикалом выбран с учетом того, что $\cos x>0$ при $|x|<\pi / 2$.

ПРИМЕР 10. Рассуждая, как и в предыдущем примере, можно показать (с учетом примера 9 из § 2 гл. IV), что
$$
\arccos ^{\prime} y=-\frac{1}{\sqrt{1-y^2}} \text { при }|y|<1 .
$$
Действительно,
$$
\arccos ^{\prime} y=\frac{1}{\cos ^{\prime} x}=-\frac{1}{\sin x}=-\frac{1}{\sqrt{1-\cos ^2 x}}=-\frac{1}{\sqrt{1-y^2}} .
$$
Знак перед радикалом выбран с учетом того, что $\sin x>0$, если $0<x<\pi$.

\end{document}